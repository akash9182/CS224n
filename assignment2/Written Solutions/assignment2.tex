
\documentclass[letter,12pt]{article}

\RequirePackage{amsmath}
\RequirePackage{amsmath,amssymb,amsthm}
\RequirePackage{tikz}
\usepackage{listings}
\usepackage{color}
\usepackage{textcomp}
\usepackage{graphicx}
\usepackage{hyperref}
\usepackage{afterpage}

\renewcommand{\lstlistlistingname}{Code Listings} 
\renewcommand{\lstlistingname}{Code Listing} 
\definecolor{gray}{gray}{0.5} 
\definecolor{key}{rgb}{0,0.5,0} 
\lstnewenvironment{python}[1][]{ 
\lstset{
language=python,
basicstyle=\ttfamily\small,
otherkeywords={1, 2, 3, 4, 5, 6, 7, 8 ,9 , 0, -, =, +, [, ], (, ), \{, \}, :, *, !},
keywordstyle=\color{blue},
stringstyle=\color{red},
showstringspaces=false,
emph={class, pass, in, for, while, if, is, elif, else, not, and, or,
def, print, exec, break, continue, return},
emphstyle=\color{black}\bfseries,
emph={[2]True, False, None, self},
emphstyle=[2]\color{key},
emph={[3]from, import, as},
emphstyle=[3]\color{blue},
upquote=true,
morecomment=[s]{"""}{"""},
commentstyle=\color{gray}\slshape,
frame=tb,
rulesepcolor=\color{blue},#1
}}{}


\usetikzlibrary{calc}
\RequirePackage{tkz-euclide}
\usetkzobj{all}
%\usepackage{minted}
%\usepackage{fontspec}
%\setsansfont{Calibri}
%\setmonofont{Consolas}
%    \begin{minted}[mathescape,
%                   linenos,
%                   numbersep=5pt,
%                   gobble=2,
%                   frame=lines,
%                   framesep=2mm]{csharp}
%      string title = "This is a Unicode π in the sky"
%      /*
%      Defined as $\pi=\lim_{n\to\infty}\frac{P_n}{d}$ where $P$ is the perimeter
%      of an $n$-sided regular polygon circumscribing a
%      circle of diameter $d$.
%      */
%      const double pi = 3.1415926535
%    \end{minted}


%\usepackage[utf8]{inputenc}
%
%% Default fixed font does not support bold face
%\DeclareFixedFont{\ttb}{T1}{txtt}{bx}{n}{12} % for bold
%\DeclareFixedFont{\ttm}{T1}{txtt}{m}{n}{12}  % for normal
%
%% Custom colors
%\usepackage{color}
%\definecolor{deepblue}{rgb}{0,0,0.5}
%\definecolor{deepred}{rgb}{0.6,0,0}
%\definecolor{deepgreen}{rgb}{0,0.5,0}
%
%\usepackage{listings}
%
%% Python style for highlighting
%\newcommand\pythonstyle{\lstset{
%language=Python,
%basicstyle=\ttm,
%otherkeywords={self},             % Add keywords here
%keywordstyle=\ttb\color{deepblue},
%emph={MyClass,__init__},          % Custom highlighting
%emphstyle=\ttb\color{deepred},    % Custom highlighting style
%stringstyle=\color{deepgreen},
%frame=tb,                         % Any extra options here
%showstringspaces=false            %
%}}
%
%
%% Python environment
%\lstnewenvironment{python}[1][]
%{
%\pythonstyle
%\lstset{#1}
%}
%{}
%
%% Python for external files
%\newcommand\pythonexternal[2][]{{
%\pythonstyle
%\lstinputlisting[#1]{#2}}}
%
%% Python for inline
%\newcommand\pythoninline[1]{{\pythonstyle\lstinline!#1!}}
%
%\begin{document}
%
%\section{``In-text'' listing highlighting}
%
%\begin{python}
%class MyClass(Yourclass):
%    def __init__(self, my, yours):
%        bla = '5 1 2 3 4'
%        print bla
%\end{python}
%
%\section{External listing highlighting}
%
%\pythonexternal{demo.py}
%
%\section{Inline highlighting}
%
%Definition \pythoninline{class MyClass} means \dots
%
%\end{document}
%    \begin{minted}{python}
%    def boring(args = None):
%    pass
%    \end{minted}

% Set the margins
%
\setlength{\textheight}{8.5in}
\setlength{\headheight}{.25in}
\setlength{\headsep}{.25in}
\setlength{\topmargin}{0in}
\setlength{\textwidth}{6.75in}
\setlength{\oddsidemargin}{0in}
\setlength{\evensidemargin}{0in}


%%%%%%%%%%%%%%%%%%%%%%%%%%%%%%%%%%%%%%%%%%%%%%%%%%%%%%%%%%%%%%%%%%%%%%%
% Macros

% Math Macros.  It would be better to use the AMS LaTeX package,
% including the Bbb fonts, but I'm showing how to get by with the most
% primitive version of LaTeX.  I follow the naming convention to begin
% user-defined macro and variable names with the prefix "my" to make it
% easier to distiguish user-defined macros from LaTeX commands.
%
\newcommand{\myN}{\hbox{N\hspace*{-.9em}I\hspace*{.4em}}}
\newcommand{\myZ}{\hbox{Z}^+}
\newcommand{\myR}{\hbox{R}}

\newcommand{\myfunction}[3]
{${#1} : {#2} \rightarrow {#3}$ }

\newcommand{\myzrfunction}[1]
{\myfunction{#1}{{\myZ}}{{\myR}}}


% Formating Macros
%

\newcommand{\myheader}[4]
{\vspace*{-0.5in}
\noindent
{#1} \hfill {#3}

\noindent
{#2} \hfill {#4}

\noindent
\rule[8pt]{\textwidth}{1pt}

\vspace{1ex} 
}  % end \myheader 

\newcommand{\myalgsheader}[0]
{\myheader{Stanford University, Department of Computer Science}
{Computer Science 224D}{Spring 2016}{Section 1}}

% Running head (goes at top of each page, beginning with page 2.
% Must precede by \pagestyle{myheadings}.
\newcommand{\myrunninghead}[2]
{\markright{{\it {#1}, {#2}}}}

\newcommand{\myrunningalgshead}[2]
{\myrunninghead{Computer Science 224D}{{#1}}}

\newcommand{\myrunninghwhead}[2]
{\myrunningalgshead{Solution to Assignment {#1}, Problem {#2}}}

\newcommand{\mytitle}[1]
{\begin{center}
{\large {\bf {#1}}}
\end{center}}

\newcommand{\myhwtitle}[3]
{\begin{center}
{\large {\bf Solution to Assignment {#1}, Problem {#2}}}\\
\medskip
{\it {#3}} % Name goes here
\end{center}}

\newcommand{\myhwintro}[3]
{\begin{center}
{\large {\bf Assignment {#1}, Problem {#2}}}\\
\medskip
{\it {#3}} % Name goes here
\end{center}}

\newcommand{\mysection}[1]
{\noindent {\bf {#1}}}

\newcommand{\solutionsAuthor}{Akash Rana}
%%%%%% Begin document with header and title %%%%%%%%%%%%%%%%%%%%%%%%%
\begin{document}


\myhwtitle{2}{1 (c)}{\solutionsAuthor}

\bigskip
Placeholder -Placeholder are nodes whose value is feed in at execution time. It is used to feed actual training example

Feed dictionaries - These are the values in batches that are input to while training 

\noindent\rule{\textwidth}{0.4pt}


\myhwtitle{2}{2 (a)}{\solutionsAuthor}
\begin{table}[!h!p]
\centering
\begin{tabular}{l|l|l|l}
  stack&buffer&new dependency&transition \\ \hline
  ... & ... & ... & ... \\
  \ [ROOT, parsed] & [this, sentence, correctly] & parsed $\rightarrow$ I & \verb|LEFT-ARC| \\
  \ [ROOT, parsed, this] & [sentence, correctly] &   & \verb|SHIFT| \\
  \ [ROOT, parsed, this, sentence] & [correctly] &  &  \verb|SHIFT| \\
  \ [ROOT, parsed, sentence] & [correctly] & sentence $\rightarrow$ this & \verb|LEFT-ARC| \\
  \ [ROOT, parsed] & [correctly] & parsed $\rightarrow$ sentence & \verb|RIGHT-ARC| \\
  \ [ROOT, parsed, correctly] & [] &  &  \verb|SHIFT| \\
  \ [ROOT, parsed] & [] & parsed $\rightarrow$ correctly &  \verb|RIGHT-ARC| \\
  \ [ROOT] & [] & ROOT $\rightarrow$ parsed & \verb|RIGHT-ARC|
\end{tabular}
\end{table}
\clearpage
\myhwtitle{2}{2 (b)}{\solutionsAuthor}
\bigskip
A sentence containing n words will be parsed in 2n ways. It will take 1 step to shift the word and 1 step to do transition and there are n words. Hence, 2n ways.
\end{document}